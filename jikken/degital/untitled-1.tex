\documentclass[a4j]{jarticle}
\usepackage[dvipdfmx]{graphicx}
\newcommand{\bs}{$\backslash$}
\renewcommand{\labelenumi}{(\roman{enumi})}
\renewcommand{\labelenumi}{(\arabic{enumi})}

\begin{document}

{\Large [概要]}\\\\
\ \ 本実験はOPアンプ(オペアンプ)の動作を理解するとともに、その基本的な使用法を習得することを目的としている。\\
\ \ OPアンプとは電圧の増幅を行う素子の一種であり、2つの入力端子(+,-)と1つの出力端子を持っている。
また、様々な用途に使用できる汎用ICとして広く利用されている。その用途の例として、比較器、増幅器、演算器などが
挙げられる。\\
\ \ これらの使用法を確認するために、OPアンプに負帰還をかけた測定回路をそれぞれの使用法に対し構成し、
入力電圧と出力電圧の関係を測定した。\\
\ \ 測定回路の電源には直流安定化電源を使用し、出力電圧の測定器としてオシロスコープを使用した。\\
\ \ それぞれの回路に対する測定結果を表にまとめ、オシロスコープの出力画面を図として表した。その結果、比較器では
入力電圧が基準電圧に近づくと出力電圧が大きく上がった。このことから入出力特性が確認できた。
増幅器では、入力電圧を徐々に上げていくと出力電圧が急激に上がり、-13.2Vまで上がるとその後は上がらない結果となった。
よって、この値がOPアンプの増幅上限と言える。また、入力を交流信号に変えてオシロスコープで波形を読み取ると、
発振器の出力が6.6Vを超えてからひずみが生まれた。
演算器では加算と減算を行った結果、全体的に誤差が大きく正しい値をとれたとは言えないが、計算したものを観測
することができた。\\
\ \ 以上の実験により、OPアンプの動作やその基本的な使用法を理解することができた。\\
\ \ 考察事項として、入力電圧がごく小さい場合とある程度増大した場合で同じ関係式が成り立つ理由の考察と、OPアンプに
正帰還をかける用途についての文献調査を行った。以下にその要点を述べる: 考察1)入力電圧がある程度増大しても
OPアンプの増幅率が非常に大きく出力電圧はほぼ変わらないため、OPアンプの+-端子間の電位差は小さいままである。
よって入力電圧がごく小さくてもある程度増大しても同じ関係式が成り立つと考えられる。 
考察2)OPアンプに正帰還をかける用途としては、持続的に正弦波的な振動を発生させる自励発振回路や入力電圧が少し
変化しても出力が切り替わらないヒステリシスコンパレータなどがある。
\newpage


\section{目的}

我々が普段利用している情報機器のなかには、様々なセンサが搭載されていることが多く、意識せずとも利用し、その恩恵を受けている。今回は扱いやすいAndroid端末を使い、センサの利用方法を学習する。

\section{原理}
\subsection{今回利用するセンサ}

今回は実機としてASUS Zenfone Live L1を用いる。搭載されているセンサには以下のようなものがある。\par
Zenfone Live L1[1]搭載センサ
\begin{itemize}
\item GPS
\item 加速度センサ
\item 電子コンパス(地磁気センサ)
\item 光センサ
\item 近接センサ
\item ジャイロスコープ
\end{itemize}




\section{まとめ}
本実験を通して、OPアンプの動作及び正帰還や負帰還などの基本的な使用法を理解できた。
また、実験において大きな誤差が生じてしまったため、今後は慎重な回路の接続確認や器具の調整を心がけていきたい。


\section{参考文献}
\begin{verbatim}
[1] ipusiron:帰還利得とループ利得
http://akademeia.info/index.php?%C8%AF%BF%B6%B2%F3%CF%A9 (2018/5/24)
[2] 熊谷正朗:オペアンプで始めるアナログ回路
http://www.mech.tohoku-gakuin.ac.jp/rde/contents/course/mechatronics/analog.html (2018/5/24)
[3] 株式会社朝日新聞社:コンパレータ
https://kotobank.jp/word/コンパレータ-3549 (2018/5/24)
[4] マルツエレック株式会社:~用語解説「オフセット電圧」~
https://www.marutsu.co.jp/pc/static/large_order/111013_opamp (2018/5/24)
[5] nteku:オフセットの調整方法
http://nteku.com/opamp/opamp-offset-chousei.aspx (2018/5/24)
[6] ローム株式会社:オペアンプとは? 回路構成
https://www.rohm.co.jp/electronics-basics/opamps/op_what2 (2018/5/24)
\end{verbatim}

\end{document}


